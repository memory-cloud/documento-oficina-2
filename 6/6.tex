\chapter{Implementa��o}

\section{Prot�tipo de Telas}

% Apresentar o prot�tipo do sistema, que consiste na interface preliminar contendo um conjunto de funcionalidades e telas. O prot�tipo � um recurso que deve ser adotado como estrat�gia para levantamento, detalhamento, valida��o de requisitos e modelagem de interface com o usu�rio (usabilidade).

% As telas do sistema podem ser criadas na pr�pria linguagem de desenvolvimento ou em qualquer outra ferramenta de desenho. Cada tela deve possuir uma descri��o do seu funcionamento, constando pelo menos o objetivo da tela e din�mica de navega��o (de onde � chamada e que outras telas pode chamar). A descri��o das telas deve registrar informa��es que possam ser consultadas para facilitar a implementa��o e a execu��o de testes, assim como a que requisitos funcionais se referem.

\section{Descri��o do C�digo}

% Descrever o sistema quanto ao c�digo gerado. Explicar a organiza��o dos arquivos, pacotes, classes ou quaisquer estruturas utilizadas no desenvolvimento do projeto, listando os componentes criados e sua estrutura. Use diagramas (Diagrama de Componentes, Diagrama de Pacotes) para ilustrar a implementa��o. 

% Descrever tamb�m conven��es e padroniza��es para coment�rios no c�digo, nomenclatura de classes, objetos, fun��es, etc. Se necess�rio, use exemplos.
